% This is "sig-alternate.tex" V2.0 May 2012
% This file should be compiled with V2.5 of "sig-alternate.cls" May 2012
%
% This example file demonstrates the use of the 'sig-alternate.cls'
% V2.5 LaTeX2e document class file. It is for those submitting
% articles to ACM Conference Proceedings WHO DO NOT WISH TO
% STRICTLY ADHERE TO THE SIGS (PUBS-BOARD-ENDORSED) STYLE.
% The 'sig-alternate.cls' file will produce a similar-looking,
% albeit, 'tighter' paper resulting in, invariably, fewer pages.
%
% ----------------------------------------------------------------------------------------------------------------
% This .tex file (and associated .cls V2.5) produces:
%       1) The Permission Statement
%       2) The Conference (location) Info information
%       3) The Copyright Line with ACM data
%       4) NO page numbers
%
% as against the acm_proc_article-sp.cls file which
% DOES NOT produce 1) thru' 3) above.
%
% Using 'sig-alternate.cls' you have control, however, from within
% the source .tex file, over both the CopyrightYear
% (defaulted to 200X) and the ACM Copyright Data
% (defaulted to X-XXXXX-XX-X/XX/XX).
% e.g.
% \CopyrightYear{2007} will cause 2007 to appear in the copyright line.
% \crdata{0-12345-67-8/90/12} will cause 0-12345-67-8/90/12 to appear in the copyright line.
%
% ---------------------------------------------------------------------------------------------------------------
% This .tex source is an example which *does* use
% the .bib file (from which the .bbl file % is produced).
% REMEMBER HOWEVER: After having produced the .bbl file,
% and prior to final submission, you *NEED* to 'insert'
% your .bbl file into your source .tex file so as to provide
% ONE 'self-contained' source file.
%
% ================= IF YOU HAVE QUESTIONS =======================
% Questions regarding the SIGS styles, SIGS policies and
% procedures, Conferences etc. should be sent to
% Adrienne Griscti (griscti@acm.org)
%
% Technical questions _only_ to
% Gerald Murray (murray@hq.acm.org)
% ===============================================================
%
% For tracking purposes - this is V2.0 - May 2012

\documentclass{sig-alternate}

\begin{document}
%
% --- Author Metadata here ---
\conferenceinfo{ICMI}{'12 Santa Monica, California USA}
%\CopyrightYear{2012} % Allows default copyright year (20XX) to be over-ridden - IF NEED BE.
%\crdata{0-12345-67-8/90/01}  % Allows default copyright data (0-89791-88-6/97/05) to be over-ridden - IF NEED BE.
% --- End of Author Metadata ---

\title{
	{\ttlit ShEMP}: A Mobile Framework for Shared Emotion, Music, and Physiology
%		\titlenote{(Produces the permission block, and copyright information). For use with SIG-ALTERNATE.CLS. Supported by ACM.}}
%	\subtitle{[Extended Abstract]
%	\titlenote{
%		A full version of this paper is available as \textit{Author's Guide to Preparing ACM SIG 
%		Proceedings Using \LaTeX$2_\epsilon$\ and BibTeX} at \texttt{www.acm.org/eaddress.htm}
%	}
}
%
% You need the command \numberofauthors to handle the 'placement
% and alignment' of the authors beneath the title.
%
% For aesthetic reasons, we recommend 'three authors at a time'
% i.e. three 'name/affiliation blocks' be placed beneath the title.
%
% NOTE: You are NOT restricted in how many 'rows' of
% "name/affiliations" may appear. We just ask that you restrict
% the number of 'columns' to three.
%
% Because of the available 'opening page real-estate'
% we ask you to refrain from putting more than six authors
% (two rows with three columns) beneath the article title.
% More than six makes the first-page appear very cluttered indeed.
%
% Use the \alignauthor commands to handle the names
% and affiliations for an 'aesthetic maximum' of six authors.
% Add names, affiliations, addresses for
% the seventh etc. author(s) as the argument for the
% \additionalauthors command.
% These 'additional authors' will be output/set for you
% without further effort on your part as the last section in
% the body of your article BEFORE References or any Appendices.

\numberofauthors{5} %  in this sample file, there are a *total*
% of EIGHT authors. SIX appear on the 'first-page' (for formatting
% reasons) and the remaining two appear in the \additionalauthors section.
%
\author{
% You can go ahead and credit any number of authors here,
% e.g. one 'row of three' or two rows (consisting of one row of three
% and a second row of one, two or three).
%
% The command \alignauthor (no curly braces needed) should
% precede each author name, affiliation/snail-mail address and
% e-mail address. Additionally, tag each line of
% affiliation/address with \affaddr, and tag the
% e-mail address with \email.
%
% 1st. author
\alignauthor Brennon Bortz\\
       \affaddr{Institute for Creativity, Arts, and Technology}\\
       \affaddr{Virginia Polytechnic Institute and State University}\\
       \affaddr{Blacksburg, VA 24060}\\
       \email{brennon@musicsensorsemotion.com}
% 2nd. author
\alignauthor Spencer Salazar\\
       \affaddr{Center for Computer Research in Music and Acoustics}\\
       \affaddr{Stanford University}\\
       \affaddr{Stanford, California 94305}\\
       \email{spencer@ccrma.stanford.edu}
% 3rd. author
\alignauthor Javier Jaimovich\\
       \affaddr{Sonic Arts Research Centre}\\
       \affaddr{Queen's University Belfast}\\
       \affaddr{Belfast, Northern Ireland}\\
       \email{javier@musicsensorsemotion.com}
\and  % use '\and' if you need 'another row' of author names
% 4th. author
\alignauthor R. Benjamin Knapp\\
       \affaddr{Institute for Creativity, Arts, and Technology}\\
       \affaddr{Virginia Polytechnic Institute and State University}\\
       \affaddr{Blacksburg, VA 24060}\\
       \email{ben@musicsensorsemotion.com}
% 5th. author
\alignauthor Ge Wang\\
       \affaddr{Center for Computer Research in Music and Acoustics}\\
       \affaddr{Stanford University}\\
       \affaddr{Stanford, California 94305}\\
       \email{ge@ccrma.stanford.edu}
}
% There's nothing stopping you putting the seventh, eighth, etc.
% author on the opening page (as the 'third row') but we ask,
% for aesthetic reasons that you place these 'additional authors'
% in the \additional authors block, viz.

%\additionalauthors{Additional authors: John Smith (The Th{\o}rv{\"a}ld Group,
%email: {\texttt{jsmith@affiliation.org}}) and Julius P.~Kumquat
%(The Kumquat Consortium, email: {\texttt{jpkumquat@consortium.net}}).}

\date{29 June 2012}
% Just remember to make sure that the TOTAL number of authors
% is the number that will appear on the first page PLUS the
% number that will appear in the \additionalauthors section.

\maketitle
\begin{abstract}
\end{abstract}

% A category with the (minimum) three required fields
%\category{H.4}{Information Systems Applications}{Miscellaneous}
%A category including the fourth, optional field follows...
%\category{D.2.8}{Software Engineering}{Metrics}[complexity measures, performance measures]

\category{H.5.3}{Information Interfaces and Presentation}{\\Group and Organization Interfaces}[Collaborative computing, Organizational design, Synchronous interaction]
\category{H.5.2}{Information Interfaces and Presentation}{User Interfaces}[Input devices and strategies]
\category{H.5.1}{Information Interfaces and Presentation}{Multimedia Information Systems}[Audio input/output]
\category{H.5.5}{Information Interfaces and Presentation}{Sound and Music Computing}
\category{C.2.4}{Computer-Communication Networks}{Distrib\-uted Systems}[Client/server]
\category{J.5}{Arts and Humanities}[Performing arts]

\terms{Algorithms, Design, Experimentation, Measurement}

\keywords{Collaborative music, group emotion, mobile computing, physiological interfaces}

\section{Introduction}
How can we measure the quality of a creative experience?  In what ways do the emotions of participants affect or are affected by creative collaboration?  Is the perception of a musical performance altered depending on whether it is experienced individually or as a member of a group?  These are among the questions under consideration by partners, including the authors, in the Social Interaction and Entrainment using Musical Performance Experimentation (SIEMPRE) project.  Here we introduce ShEMP--a software framework through which we can explore these questions in greater depth.  ShEMP, a mobile framework for Shared Emotion, Music, and Physiology, in conjunction with MobileMuse, an unobtrusive sensor package for mobile physiological signal acquisition, leverage the distributed yet locative properties of mobile devices to allow the design of ecological experiments outside of the laboratory to investigate collaborative creativity and shared experience of musical performances.  This paper provides a brief introduction to several notable advances made in recent and current SIEMPRE experiments that have been particularly motivating to the development of ShEMP.  This is followed by an overview of the design of ShEMP and a discussion of the suite of technologies it employs.  We then elaborate on an initial battery of experiments to be executed presently, for which the framework was designed.

\section{Background}
For the last two years, the Social Interaction and Entrainment using Music Performance Experimentation (SIEMPRE) project has focused on measuring interpersonal creative interaction on the backdrop of music performance.  Three parts of this interaction are of particular interest to SIEMPRE: co-creation, emotional contagion, and entrainment.  The experiments designed and executed thus far have focused on these the experience of musical performance and experience in the following interconnected areas:

\begin{itemize}
	\item Listener/listener interactions
	\item Performer/performer interactions
	\item Conductor/performer interactions
	\item Audience experience
	\item Music
\end{itemize}

\section{Related Work}

\section{Recent Results and Motivation}

\section{Framework Design}

\section{Proposed Experiment}

\section{Potential Issues}

\section{Conclusions}

%\end{document}  % This is where a 'short' article might terminate

%ACKNOWLEDGMENTS are optional
\section{Acknowledgments}

%
% The following two commands are all you need in the
% initial runs of your .tex file to
% produce the bibliography for the citations in your paper.
\bibliographystyle{abbrv}
\bibliography{ICMI_Paper_References}
% You must have a proper ".bib" file
%  and remember to run:
% latex bibtex latex latex
% to resolve all references
%
% ACM needs 'a single self-contained file'!
%
%APPENDICES are optional
%\balancecolumns
%\balancecolumns % GM June 2007
% That's all folks!
\end{document}
